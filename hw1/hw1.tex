\documentclass[a4paper]{article}
\usepackage{ctex}

\title{\heiti 《数字图像处理》第1周作业}
\author{陈彦旭\;无24}
\date{\today}
\begin{document}

\maketitle

请至少举2个事例说明生活中的“数字图像处理”案例(非课堂例子),说明内容包括涉及图像的类型、应用场合和实际意义等。


案例1:
应用软件中的滤镜和美颜功能,例如抖音等平台的特效和美颜。涉及的图像类型:照片、实况照片、视频。应用场合:社交媒体中用户上传照片、视频时可以自己选择加入特效、美颜、滤镜等修改作品效果,直播中也可以实时使用美颜、虚拟场景等改进直播间效果。实际意义:自动美颜等功能利用数字图像处理算法识别人脸区域,通过美白、磨皮、瘦脸等处理。也能够自动识别照片中的内容和场景,自动调节亮度、对比度、饱和度等参数,增强图片效果,提高用户上传照片、视频的质量,降低用户使用成本,提高用户体验。


案例2:
农田检测和病虫害预防。设计的图片类型:无人机或卫星拍摄的农田照片。应用场合:在农业中利用无人机和卫星拍摄农田照片,利用数字图像处理技术分析图像,检测农作物生长状况,识别病虫害区域,及时预防。实际意义:为农业耕种提供精确的针对性建议,提高农业生产效率、农药和化肥使用效率,提高农产品质量,帮助农民增收。


\end{document}